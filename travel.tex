\section{Travel}

\subsection{Airports and connections}

\subsubsection{Frankfurt International Airport}

According to wikipedia there are "107 airlines flying to 275 destinations in 111 countries from Frankfurt Airport, with approximately 1,365 flights per day". Frankfurt is one of the major long-haul hubs of Europe. Karlsruhe, only a stone-throw away is hence accessible from all around the world.

\url{http://www.frankfurt-airport.de}

From Frankfurt Airport, Karlsruhe can be reached by train or by bus. While the shortest, but also most expensive option is by train (ICE 1h05, 40 Euro), bus trips take about 2h00 (from 1h45 to 3h00) for about a fourth of the price.

We recommend \url{https://www.busliniensuche.de} to find the best option.

\iffalse
\subsubsection{Frankfurt Hahn}

Various airlines are connecting Frankfurt Hahn to about 30 destinations in Europe. Most of these airlines offer fairly cheap fares (Ryanair, Easyjet etc.). The airport is only accessible from Karlsruhe by bus over Mannheim or Frankfurt. The trip takes about 4 hours and costs roughly 25 euros.

\url{https://www.hahn-airport.de}
\fi

\subsubsection{Stuttgart Airport}

About 55 airlines are flying to over a 100 destinations from Stuttgart airport. The trip from Stuttgart Airport to Karlsruhe by train takes 1h35 and costs 25.50 euros. Postbus, as well as other bus lines, have offers as cheap as 5 euros. Bus trips, leaving directly at the airport, take about an hour. 

\url{http://www.flughafen-stuttgart.de}

\subsubsection{Karlsruhe Airport/Baden-Baden}

Airlines are flying to about 60 destinations around the world during summer times. The Hahn-Express, the same bus line connecting Karlsruhe to Frankfurt Hahn, allows airport access within half an hour.
 
\url{https://www.baden-airpark.de}

\subsubsection{Strasbourg}

As already known per the GUADEC in 2014.
Karlsruhe can be reached via TGV or local train in about an hour.


\subsection{Bus}

Karlsruhe has a big bus terminal with direct connections to other German cities, to France, Norway, Sweden, Belgium, The UK, The Netherland, Czech Republic etc.

We recommend \url{https://www.busliniensuche.de} to find the best option.

\subsection{Train}

Karlsruhe has obviously a train station with direct connections to other German cities, to France, Norway, Sweden, Belgium, The UK, The Netherland, Czech Republic etc.

We recommend \url{https://www.db.de} to find the best option.

A team member of the organizing committee is an expert in finding the best and cheapest travel option and will be delighted to assist you in finding your way to Karlsruhe for GUADEC 2016!

\subsection{Car and Parking}

If you want to park your car in the inner city of Karlsruhe, a special ticket for 5 euros is required. 

Details can be found at:
\url{http://www.karlsruhe.de/b3/natur_und_umwelt/umweltschutz/luftreinhaltung/umweltzonen.de}

If you need assistance, the organizing committee will assist you.

\subsection{Local public transport}



The local public transport system (KVV) is a tight network of buses and trams connecting not only the different part within the city, but anchoring the city in the wider context of Baden-Würthemberg. The organizing committee is committed to investigate the possibility to get special day passes for all participants for the duration of the event. Alternatively there are day passes for 5 people for 9euros.

Karlsruhe is a compact city. The trip from the train station (southern end of Karlsruhe) to the venue (north east end) for instance takes just under 35 Minutes.



Notes Benjamin:
 * This is *not* the document describing travel options to attendees.
 * Umweltplakete: does not belong into bid; for website just provide info how one can park outside without entering (i.e. Waldparkplatz)
 * Car: If at all, then reason why reachable for a lot of people using this method (KA is central in Europe)
 * need to think about how to write the text, thinking of a short summary sentence and the important details in a tabular format. Also consider layout for document!
