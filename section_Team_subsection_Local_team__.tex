
\section{Team}

\subsection{Local team}
The local core team is composed of three people.

Benjamin Berg from the GNOME community is joined by Markus Mohardt, from the Collabora community, and by Manuel Loesch from the FZI to ensure that the overall concept is put into place, that the technical content of the conference is appealing and challenging and that social events will contribute to create a unique experience.

\subsection{Support team}

The core team is supported by Tobias Mueller (Frankfurt), Steff Walters (Ettlingen) and Alexandre Frank (Strasbourg), as well as by Moira Schuler and Joykie Walters. Sponsoring, communication, finance, staff management and contribution to content will be ensured by the support team which is living in reachable distance from the actual venue. 

\subsection{Local support}

Further support is provided by Benni Fury and Dominik Hufnagel who will ensure close connections to university students and PhD students at the SAP Research Institute or the FZI. The plan is to tap into the student pool for to reach for volunteers and participants to create a diversity of backgrounds during the conference. 

\subsubsection{GNOME community}

XXXX


\subsubsection{Other groups and communities}

Three subsection of KALUG, the Linux User Group in Karlsruhe, will be contacted to ensure close collaboration and favor attendance of participants from other communities. 

\subsubsection{ Government and industry}

Karlsruhe is not only a student city; it is also a hub for IT companies. Every year, events such as LEARNTEC and initiatives such as KAIT-SI or MEKA are hosted in Karlsruhe, bringing together professionals from the area of security, mobile computing and design. The committee is already working on fostering connections to these companies to ensure contributions and sponsoring. We believe that hosting GUADEC 2016 in Karlsruhe can contribute to anchoring GNOME in a sustainable way in the wider network of a fast-growing IT-cluster in Germany which has its central point in Baden-Würthemberg. We further commit to closely interact with the local government, administration and tourism board to ensure that as many entities as possible are aware of the unique opportunity that an event such as GUADEC is providing to members of the GNOME community, but also to the wider public.  
