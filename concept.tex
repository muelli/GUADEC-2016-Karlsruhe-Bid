\section{Dates}

The traditional concept of two core days with keynotes and other 
presentation, followed by a two-day hackfest was proven to be a winning 
formula. We intend to keep those basics and add a little twist.

We plan to run a pre-event on Friday, especially designed for 
students in computer sciences degrees from the various universities in 
Karlsruhe. The pre-event includes a general introduction to free and 
open source software and a more specific overview over the GNOME 
project and is followed by a few hands-on sessions. As two of the 
members of the organizing team are teaching at DHBW, one of the 
universities, we will be able to seamlessly integrate some particular 
aspects of the GNOME projects with topics taught at university. 
Ideally, we would like to involve our GSoC students as much as 
possible, especially during the hands-on sessions, in order to lower 
the initial barriers by creating a stimulating environment where the 
students will feel that they are working with peers.

\iffalse
Furthermore, we plan to host the Sunday lecture session at ZKM, where 
the end of the day can then be spent discovering the incredible variety 
of art exhibitions and challenging each other at various games.
\fi

In summary, the event will be scheduled as follows:

Friday:
\begin{itemize}
\item Pre-event for students
\item Registration
\item Social event at AKK or Z10
\end{itemize}

Saturday:
\begin{itemize}
\item  Core day I
\item  Social event at XXX
\end{itemize}

Sunday:
\begin{itemize}
\item  Core day II at ZKM
\item  AGM
\item  GSoC Student Talks
\end{itemize}

Monday:
\begin{itemize}
\item  Hackfest day I
\item  Visit of the Hoepfner Brewery
\end{itemize}

Tuesday:
\begin{itemize}
\item  Hackfest day II
\item  Picnic in the “Schlossgarten”
\end{itemize}


Due to availability of the venues, we target the beginning or mid
of August.
In fact, we are planning for CW32-33, so from 2015-08-10 until 2015-08-17.
