
\newpage

\begin{tikzpicture}[remember picture,overlay]
  \begin{scope}[on background layer]
    \node[anchor=north west,outer sep=0,inner sep=0] (img) at (current page.north west) {\includegraphics[width=21cm,clip,trim=0 250 0 0]{images/city/naturkundemuseum-.jpg}};
  \end{scope}
  \node[anchor=south west,color=black,xshift=1ex,yshift=1ex] (label) at (img.south west) {\imgtitle{Günter Josef Radig}{State Museum of Natural History}{CC BY-NC-SA 2.0, cropped, \url{http://ka.stadtwiki.net/Datei:Naturkundemuseum_Karlsruhe.jpg}}};
  \begin{scope}[on background layer]
    \node[fit=(label),inner sep=0,outer sep=0,opacity=0.6,fill=white,rounded corners] {};
  \end{scope}
\end{tikzpicture}

\vspace*{9.5cm}


\section{Team}

The team is currently composed of four people:
Benjamin Berg, Markus Mohrhard, Moira Schuler, and Tobias Mueller.

We have commitments from more people both local and remote.
We will need more help, though, and we have already discussed
what the roles and responsibilities are that need to be distributed.

Local GNOME communities are hard to find in Germany, these days,
so there is no local GNOME user group in Karlsruhe.
But there are several IT and IT friendly institutions such as the
the local CCC branch called Entropia, a FabLab, and several LUGs.
Some of those have already organized an event like GUADEC and
we hope to make use of their knowledge.

As for the legal structure supporting the event,
we are considering working together with a student association at the
university as this would simplify access to the university's resources
and also help us to tap into the student's pool more easily.
We are also consider founding a German non-profit to handle the
local finances.
