\section{Introductory paragraph about the location}

Let's start out with some trivialities that might convince you all, that hosting GUADEC in Karlsruhe in 2016 is a brilliant idea. Karlsruhe is Germany's sunniest city.  Karlsruhe is situated in the Southern part of Germany, hence not only easily accessible from other European Countries, but also from abroad as it is close enough to Frankfurt international airport and Stuttgart. Karlsruhe combines a strong sense for science and technology - think KIT, FZI and Frauenhofer Institute - with a penchant for creativity and design - think ZKM. Karlsruhe is a typical "student city" with affordable accommodation and cheap meals. The city hosts 4 higher education institutes offering computer science related degrees. As we plan a particular student pre-event to tap into this student pool, the potential of growing the GNOME community is guaranteed.  

Karlsruhe is the perfect place to host GUADEC 2016. 

\section{Venue}


We plan on hosting GUADEC 2016 at DHBW on Friday, Saturday, Monday and Tuesday and to spend Sunday at ZKM.

DHBW offers exactly what we are looking for: A large room for keynotes and some smaller rooms for parallel sessions and rooms for the BoFs during the two-day hackfest. The rooms are all perfectly equipped with all one needs and the venue has a lobby which we intend to transform into a lounge for casual gatherings. 

ZKM offers large auditorium which would be perfect for hosting the AGM and the GSoC Student Talks. There are also a few smaller rooms available for parallel sessions. As one cannot visit Karlsruhe without spending some time at ZKM, the end of the day on Sunday will be dedicated to visit this particularly interactive museum.

DHBW is located in the northern part of Karlsruhe. It is in walking distance of the inner city. It is also in walking distance of the accommodation that we are suggesting. We are nonetheless in discussion with the local tourism office to arrange for reduced fares for public transport to render the access to the venue even easier. 

In the unlikely event of things not going the way they should, alternative options such as hosting the event entirely at ZKM or relocating to the Karlshochschule, right in the heart of the inner city, have already been considered and can be activated if required. However, additional costs might occur, especially if we spend the entire time at ZKM. 

\section{Travel}

\subsection{Airports and connections}

\subsubsection{Frankfurt International Airport}

According to wikipedia there are "107 airlines flying to 275 destinations in 111 countries from Frankfurt Airport, with approximately 1,365 flights per day". Frankfurt is one of the major long-haul hubs of Europe. Karlsruhe, only a stone-throw away is hence accessible from all around the world.

http://www.frankfurt-airport.de/content/frankfurt_airport/de.html

From Frankfurt Airport, Karlsruhe can be reached by train or by bus. While the shortest, but also most expensive option is by train (ICE 1h05, 40 Euro), bus trips take about 2h00 (from 1h45 to 3h00) for about a fourth of the price.

We recommend https://www.busliniensuche.de/blog/ to find the best option.

\subsubsection{Frankfurt Hahn}

Various airlines are connecting Frankfurt Hahn to about 30 destinations in Europe. Most of these airlines offer fairly cheap fares (Ryanair, Easyjet etc.). The airport is only accessible from Karlsruhe by bus from over Mannheim or Frankfurt. The trip takes about 4 hours and costs roughly 25 euros.

https://www.hahn-airport.de/

\subsubsection{Stuttgart Airport}

About 55 airlines are flying to over a 100 destinations from Stuttgart airport. The trip from Stuttgart Airport to Karlsruhe by train takes 1h35 and costs 25.50 euros. Postbus, as well as other bus lines, have offers as cheap as 5 euros. Bus trips, leaving directly at the airport, take about an hour. 

http://www.flughafen-stuttgart.de/

\subsubsection{Karlsruhe Airport / Baden-Baden}

Airlines are flying to about 60 destinations around the world during summer times. The Hahn-Express, the same bus line connecting Karlsruhe to Frankfurt Hahn, allows airport access within half an hour.
 
https://www.baden-airpark.de/startseite.html

\subsection{Bus}

Karlsruhe has a big bus terminal with direct connections to other German cities, to France, Norway, Sweden, Belgium, The UK, The Netherland, Czech Republic etc.

We recommend https://www.busliniensuche.de/blog/ to find the best option.

\subsection{Train}

Karlsruhe has a big bus terminal with direct connections to other German cities, to France, Norway, Sweden, Belgium, The UK, The Netherland, Czech Republic etc.

We recommend https://www.db.de to find the best option.

A team member of the organizing committee is an expert in finding the best and cheapest travel option and will be delighted to assist you in finding your way to Karlsruhe for GUADEC 2016!

\subsection{Car and Parking}

If you want to park your car in the inner city of Karlsruhe, a special ticket for 5 euros is required. 

Details can be found at:
http://www.karlsruhe.de/b3/natur_und_umwelt/umweltschutz/luftreinhaltung/umweltzonen.de

If you need assistance, the organizing committee will assist you.

\subsection{Local public transport}



The local public transport system (KVV) is a tight network of buses and trams connecting not only the different part within the city, but anchoring the city in the wider context of Baden-Würthemberg. The organizing committee is committed to investigate the possibility to get special day passes for all participants for the duration of the event. Alternatively there are day passes for 5 people for 9euros.

Karlsruhe is a compact city. The trip from the train station (southern end of Karlsruhe) to the venue (north east end) for instance takes just under 35 Minutes.

\section{Accommodation}

A&O, the hostel situated next to the train station (and hence about 35 minutes by foot from the venue) costs about 10 for shared accommodation. Private dorms are also available for the cost of 25 euros. 

As we would like to keep everyone located in the same area, we would like to host keynote speakers rooms at the hotel Ibis located just on the other side of the train station. 
Alternatively, B&B, another hostel, is offering rooms for about 25 euros per person sharing if need be. 


\section{Dining}

\section{Breakfast}
There are little German bakeries all over the city where you can get breakfast (savory or sweet pastries and coffee or tea) for about 4 euros. However, breakfast is included in room fares at the Ibis hotel and A&O, as well B&B have common dining rooms and kitchen where everyone can prepare their usual breakfast.
 
\subsection{Lunch}
The university canteen offers meals for about 3 euros. Otherwise, all type of restautrants (kebabs, thai,chinese, italian etc.) are available all over the inner city. On average, one should budget about 6 euros per meal.


\subsection{Dinner}
We intend to organize a picnic in the “Schlosspark” on Tuesday, as well as a BBQ on Friday. For the other evenings, we would suggest to have dinner at one of the three “Hammer” options (Kippe, Emaille, Blue) for under a fiver, including a glass of beer! Of course, other options are available in this city which is known for its student-friendly prices!


As you can see, accessibility of the location, prices of accommodation and meals, as well as the overall concept elaborated by the organizing committee are all speaking in favor of hosting GUADEC in August 2016 in the sunny city of Karlsruhe! We are looking forward to welcoming you!

\section{Budget}
\section{Bibliograpyh and credits}